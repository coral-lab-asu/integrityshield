\documentclass[12pt]{article}
\usepackage{graphicx}
\usepackage{amsmath, amssymb}
\usepackage[margin=1in]{geometry}
\usepackage[utf8]{inputenc}

\graphicspath{{./ASU_Physics Quiz_assets/}}

\begin{document}

\begin{center}
    \includegraphics[width=0.8\textwidth]{page1_img1.png}
\end{center}

\begin{tabular}{l r}
    Term: Fall 2023 & Subject: Physics (PHY) \hspace{1cm} Course Number: 150 \\
\end{tabular}

\vspace{0.5cm}
\hrule
\vspace{0.5cm}

\begin{center}
    \textbf{\large Weekly Quiz 1 (10 points)} \\
    \textcolor{red}{\textbf{Due on Friday, September 1\textsuperscript{st} at 11:59 pm}}
\end{center}

\vspace{0.5cm}

\begin{enumerate}
    \item The unit of electrical resistance is:
    \begin{enumerate}
        \item Ampere
        \item Ohm
        \item Coulomb
        \item Volt
    \end{enumerate}

    \item What is the SI unit of power?
    \begin{enumerate}
        \item Watt
        \item Joule
        \item Newton
        \item Pascal
    \end{enumerate}

    \item What phenomenon in a prism causes the splitting of white light into its constituent colors?
    \begin{enumerate}
        \item Dispersion
        \item Refraction
        \item Diffraction
        \item Polarization
    \end{enumerate}

    \item What is Newton’s second law of Motion?
    \begin{enumerate}
        \item The acceleration of an object is directly proportional to the net force acting on it and inversely proportional to its mass.
        \item If two systems are in thermal equilibrium with a third system, then they are also in thermal equilibrium with each other.
        \item For every action, there is an equal and opposite reaction. When one object exerts a force on a second object, the second object exerts an equal and opposite force on the first.
        \item An object at rest will stay at rest, and an object in motion will stay in motion with the same velocity (speed and direction) unless acted upon by an unbalanced external force.
    \end{enumerate}

    \item Which variable must remain constant for Boyle’s law to hold?
    \begin{enumerate}
        \item Pressure
        \item Temperature
        \item Volume
        \item Amount of gas
    \end{enumerate}

    \item In a Hooke’s-law spring experiment, which graph is linear for small extensions?
    \begin{enumerate}
        \item Force vs extension
        \item Energy vs extension
        \item Force vs time
        \item Displacement vs time
    \end{enumerate}

    \item Components connected so that the same current flows through each are in:
    \begin{enumerate}
        \item Series
        \item Parallel
        \item Resonant
        \item Open
    \end{enumerate}

    \item At constant pressure, the heat added to a system equals the change in:
    \begin{enumerate}
        \item Enthalpy
        \item Internal Energy
        \item Gibbs free energy
        \item Helmholtz free energy
    \end{enumerate}

    \item Which law describes the magnitude of the gravitational force between two point masses?
    \begin{enumerate}
        \item Newton’s law of gravitation
        \item Coulomb’s law
        \item Ampere’s law
        \item Faraday’s law
    \end{enumerate}

    \item The energy possessed by a body due to its motion is called:
    \begin{enumerate}
        \item Potential Energy
        \item Kinetic Energy
        \item Mechanical Energy
        \item Chemical Energy
    \end{enumerate}
\end{enumerate}

\end{document}