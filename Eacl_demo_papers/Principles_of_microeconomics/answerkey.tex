\documentclass[11pt]{article}

\usepackage[margin=1in]{geometry}
\usepackage{enumitem}
\usepackage{amsmath, amssymb}
\usepackage{hyperref}

\setlist[enumerate]{itemsep=0.8em, topsep=0.8em}

\begin{document}

\begin{center}
  {\Large \textbf{Principles of Microeconomics}}\\[0.25em]
  {\large \textbf{Supply, Demand, and Market Equilibrium Quiz -- Answer Key}}\\[0.5em]
\end{center}

\begin{enumerate}[label=\textbf{\arabic*.}]
  \item \textbf{(B) A movement along the demand curve.} Price changes cause movements along the curve. Only non-price factors (income, preferences, etc.) shift the entire curve.

  \item \textbf{(B) Increase demand for the other good.} Substitutes are goods that can replace each other. If coffee prices rise, demand for tea (a substitute) increases as consumers switch.

  \item \textbf{(B) A shortage.} A binding price ceiling prevents the market from reaching equilibrium. At the artificially low price, quantity demanded exceeds quantity supplied, creating a shortage.

  \item \textbf{(B) The responsiveness of quantity demanded to price changes.} Elasticity is calculated as the percentage change in quantity demanded divided by the percentage change in price.

  \item \textbf{(B) Increase total revenue.} When demand is inelastic ($|E_d| < 1$), the percentage decrease in quantity is smaller than the percentage increase in price, so total revenue rises.

  \item \textbf{False.} This is true only for normal goods. For inferior goods (e.g., instant noodles, public transit), increased income shifts demand to the left as consumers substitute toward higher-quality alternatives.

  \item \textbf{True.} Market equilibrium occurs at the price where $Q_s = Q_d$. At this point, there is no tendency for price to change, and the market clears.

  \item \textbf{False.} Technological improvements reduce production costs, making it profitable to supply more at every price. This shifts the supply curve to the right (outward), not left.

  \item \textbf{Key distinction:} A change in quantity demanded is a movement along a fixed demand curve caused by a price change. A change in demand is a shift of the entire curve caused by non-price factors.
  
  \textbf{Factors shifting demand:}
  
  (1) \textit{Consumer income:} Rising incomes increase demand for normal goods. Example: Economic growth increases demand for automobiles.
  
  (2) \textit{Prices of related goods:} Higher prices of substitutes increase demand for a good. Example: Rising beef prices increase demand for chicken.
  
  (3) \textit{Consumer preferences:} Changing tastes shift demand. Example: Health awareness has shifted demand toward organic foods.
  
  (4) \textit{Expectations:} Expected future price increases raise current demand. Example: Anticipated tariffs cause consumers to buy imported goods now.
  
  (5) \textit{Population/demographics:} More consumers increase market demand. Example: Aging populations increase demand for healthcare.

  \item \textbf{Mechanism:} A minimum wage above equilibrium creates a price floor in the labor market. At the higher wage, quantity of labor supplied (workers seeking jobs) exceeds quantity demanded (jobs offered), resulting in unemployment (surplus of labor).
  
  \textbf{Benefits:}
  \begin{itemize}
    \item Workers who retain employment earn higher wages
    \item Reduced poverty and income inequality for employed workers
    \item Increased consumer spending from low-wage workers with high marginal propensity to consume
    \item Reduced worker turnover and training costs for firms
  \end{itemize}
  
  \textbf{Drawbacks:}
  \begin{itemize}
    \item Unemployment among low-skilled workers priced out of the market
    \item Reduced hours for some workers
    \item Higher prices passed to consumers
    \item Potential business closures in labor-intensive industries
  \end{itemize}
  
  \textbf{Conclusion:} The net effect depends on elasticity of labor demand. Empirical evidence shows moderate minimum wage increases have smaller disemployment effects than traditional theory predicts.

\end{enumerate}

\end{document}