\documentclass[11pt]{article}

\usepackage[margin=1in]{geometry}
\usepackage{enumitem}
\usepackage{amsmath, amssymb}
\usepackage{hyperref}

\setlist[enumerate]{itemsep=0.8em, topsep=0.8em}

\begin{document}

\begin{center}
  {\Large \textbf{Introduction to Psychology}}\\[0.25em]
  {\large \textbf{Learning and Memory Quiz -- Answer Key}}\\[0.5em]
\end{center}

\begin{enumerate}[label=\textbf{\arabic*.}]
  \item \textbf{(C) Neutral stimulus.} Before conditioning, the bell did not elicit salivation. Only after repeated pairing with food (unconditioned stimulus) did it become a conditioned stimulus.

  \item \textbf{(B) Variable ratio.} Variable ratio schedules (reinforcement after an unpredictable number of responses) produce high, steady response rates and are resistant to extinction. This is why gambling is so addictive.

  \item \textbf{(A) Items at the beginning and end of a list better than middle items.} This comprises the primacy effect (better recall of early items) and recency effect (better recall of recent items).

  \item \textbf{(B) Short-term memory.} George Miller's famous paper established that short-term memory holds approximately 7 $\pm$ 2 chunks of information.

  \item \textbf{(B) Loss of memories formed before the trauma.} Retrograde amnesia affects past memories, while anterograde amnesia (option A) affects formation of new memories.

  \item \textbf{False.} Negative reinforcement increases behavior by removing an unpleasant stimulus. Punishment decreases behavior by applying an unpleasant stimulus or removing a pleasant one.

  \item \textbf{True.} The hippocampus is critical for consolidating declarative (explicit) memories—facts and events. Damage to the hippocampus impairs new memory formation, as demonstrated in patient H.M.

  \item \textbf{False.} Research by Loftus and others demonstrates that memory is reconstructive and susceptible to suggestion, misinformation, and false memories. Eyewitness testimony is often unreliable.

  \item \textbf{Classical Conditioning (Pavlov):}
  \begin{itemize}
    \item Learning through association between stimuli
    \item Involves involuntary, reflexive responses
    \item Key concepts: unconditioned stimulus/response, conditioned stimulus/response, extinction, generalization
    \item Applications: advertising (pairing products with positive emotions), treating phobias (systematic desensitization)
  \end{itemize}
  
  \textbf{Operant Conditioning (Skinner):}
  \begin{itemize}
    \item Learning through consequences of voluntary behavior
    \item Involves reinforcement (increases behavior) and punishment (decreases behavior)
    \item Key concepts: positive/negative reinforcement, positive/negative punishment, shaping, schedules of reinforcement
    \item Applications: classroom management, animal training, behavioral therapy, workplace incentives
  \end{itemize}
  
  \textbf{Key differences:} Classical involves passive learning of associations; operant involves active behavior modified by consequences. Classical deals with reflexive responses; operant with voluntary behaviors.

  \item \textbf{The Model:} Atkinson and Shiffrin (1968) proposed three distinct memory stores: (1) Sensory memory—brief storage of sensory information (iconic, echoic); (2) Short-term memory—limited capacity (7$\pm$2 items), short duration (15-30 seconds without rehearsal); (3) Long-term memory—unlimited capacity, potentially permanent storage.
  
  \textbf{Strengths:} Influential framework; supported by case studies (H.M. showed intact STM but impaired LTM formation); primacy/recency effects support distinction between stores; neuroimaging shows different brain regions involved.
  
  \textbf{Criticisms:}
  
  (1) \textit{Oversimplification of STM:} Baddeley's Working Memory Model proposes multiple components (phonological loop, visuospatial sketchpad, central executive, episodic buffer) rather than a unitary STM.
  
  (2) \textit{LTM not unitary:} Evidence distinguishes declarative (explicit: semantic, episodic) and procedural (implicit) memory, with different neural substrates.
  
  (3) \textit{Rehearsal not necessary:} Flashbulb memories and incidental learning show information can enter LTM without deliberate rehearsal.

\end{enumerate}

\end{document}