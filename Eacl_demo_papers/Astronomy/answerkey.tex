\documentclass[11pt]{article}

\usepackage[margin=1in]{geometry}
\usepackage{enumitem}
\usepackage{amsmath, amssymb}
\usepackage{hyperref}

\setlist[enumerate]{itemsep=0.8em, topsep=0.8em}

\begin{document}

\begin{center}
  {\Large \textbf{Introduction to Astronomy}}\\[0.25em]
  {\large \textbf{Stellar Evolution and Cosmology Quiz -- Answer Key}}\\[0.5em]
\end{center}

\begin{enumerate}[label=\textbf{\arabic*.}]
  \item \textbf{(C) Hydrogen fusion in the core.} Main-sequence stars convert hydrogen to helium via nuclear fusion (proton-proton chain or CNO cycle), releasing enormous energy through mass-energy conversion ($E=mc^2$).

  \item \textbf{(C) Luminosity and surface temperature.} The H-R diagram plots luminosity (y-axis) versus temperature/spectral type (x-axis). Stars cluster in characteristic regions based on their evolutionary stage.

  \item \textbf{(C) White dwarf.} Sun-like stars shed outer layers as planetary nebulae, leaving behind a degenerate carbon-oxygen core—a white dwarf supported by electron degeneracy pressure.

  \item \textbf{(C) Galaxies recede at velocities proportional to their distance.} $v = H_0 \times d$ where $H_0$ is Hubble's constant (~70 km/s/Mpc). This implies universal expansion.

  \item \textbf{(B) The early hot, dense universe.} The CMB is relic radiation from ~380,000 years after the Big Bang when the universe cooled enough for atoms to form and photons to travel freely.

  \item \textbf{True.} When hydrogen exhausts in the core, shell burning expands the outer layers, increasing luminosity and radius while the surface cools (redder color).

  \item \textbf{False.} Black holes don't emit light (nothing escapes the event horizon). They're detected indirectly through gravitational effects, accretion disk radiation, and gravitational waves.

  \item \textbf{True.} Estimates suggest ~10\textsuperscript{22}-10\textsuperscript{24} stars in the observable universe, comparable to estimates of sand grains on Earth (~10\textsuperscript{22}-10\textsuperscript{24}).

  \item \textbf{Massive Star Life Cycle:}
  
  \textbf{Formation:}
  \begin{itemize}
    \item Gravitational collapse of giant molecular cloud
    \item Protostar forms, heats up
    \item Nuclear fusion ignites when core reaches ~10 million K
  \end{itemize}
  
  \textbf{Main Sequence (Hydrogen Burning):}
  \begin{itemize}
    \item CNO cycle dominates (more efficient at high temperatures)
    \item Duration: millions of years (shorter than low-mass stars)
    \item Core converts H → He
  \end{itemize}
  
  \textbf{Post-Main Sequence Evolution:}
  \begin{itemize}
    \item Helium burning: He → C, O (triple-alpha process)
    \item Carbon burning: C → Ne, Mg
    \item Neon burning: Ne → O, Mg
    \item Oxygen burning: O → Si, S
    \item Silicon burning: Si → Fe (iron)
    \item Onion-layer structure develops
  \end{itemize}
  
  \textbf{Core Collapse and Supernova:}
  \begin{itemize}
    \item Iron cannot fuse to release energy (endothermic)
    \item Core collapses in milliseconds
    \item Rebound creates Type II supernova
    \item Heavy elements (beyond iron) created via r-process
  \end{itemize}
  
  \textbf{Final States:}
  \begin{itemize}
    \item 8-25 $M_\odot$: Neutron star (supported by neutron degeneracy)
    \item $>$25 $M_\odot$: Black hole (gravity overcomes all pressure)
  \end{itemize}

  \item \textbf{Evidence for Big Bang Theory:}
  
  \textbf{1. Cosmic Microwave Background (CMB):}
  \begin{itemize}
    \item Discovered 1965 by Penzias and Wilson
    \item Nearly uniform 2.725 K blackbody radiation in all directions
    \item Predicted by Big Bang: remnant heat from early universe
    \item Tiny fluctuations (~1 part in 100,000) seed large-scale structure
  \end{itemize}
  
  \textbf{2. Hubble's Law and Expansion:}
  \begin{itemize}
    \item Galaxies exhibit redshift proportional to distance
    \item Universe is expanding; running time backward implies denser, hotter past
    \item Extrapolation gives age ~13.8 billion years
    \item Accelerating expansion discovered 1998 (dark energy)
  \end{itemize}
  
  \textbf{3. Primordial Nucleosynthesis:}
  \begin{itemize}
    \item Big Bang predicts specific light element abundances
    \item ~75\% H, ~25\% He, traces of D, Li formed in first minutes
    \item Observed abundances match predictions precisely
    \item Cannot be explained by stellar nucleosynthesis alone
  \end{itemize}
  
  \textbf{Additional evidence:}
  \begin{itemize}
    \item Large-scale structure consistent with early density fluctuations
    \item Time dilation in distant supernovae
    \item No objects older than ~13.8 billion years observed
  \end{itemize}

\end{enumerate}

\end{document}