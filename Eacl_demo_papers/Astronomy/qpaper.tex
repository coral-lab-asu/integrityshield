\documentclass[11pt]{article}

\usepackage[margin=1in]{geometry}
\usepackage{enumitem}
\usepackage{amsmath, amssymb}
\usepackage{hyperref}

\setlist[enumerate]{itemsep=0.8em, topsep=0.8em}
\setlist[itemize]{itemsep=0.3em, topsep=0.3em}

\begin{document}

\begin{center}
  {\Large \textbf{Introduction to Astronomy}}\\[0.25em]
  {\large \textbf{Stellar Evolution and Cosmology Quiz}}\\[0.5em]
\end{center}

\noindent\textbf{Instructions:}
\begin{itemize}
  \item Answer all questions.
  \item For Questions 1--5, choose the best option.
  \item For Questions 6--8, mark True or False.
  \item For Questions 9--10, write detailed answers with scientific explanations.
\end{itemize}

\vspace{0.5em}

\begin{enumerate}[label=\textbf{\arabic*.}]
  \item The primary energy source for main-sequence stars like our Sun is:
  \begin{enumerate}[label=(\Alph*)]
    \item Gravitational contraction
    \item Nuclear fission
    \item Hydrogen fusion in the core
    \item Chemical combustion
  \end{enumerate}

  \item A star's position on the Hertzsprung-Russell diagram is determined by its:
  \begin{enumerate}[label=(\Alph*)]
    \item Distance from Earth
    \item Age and composition
    \item Luminosity and surface temperature
    \item Mass and rotation rate
  \end{enumerate}

  \item What is the ultimate fate of a star with approximately one solar mass?
  \begin{enumerate}[label=(\Alph*)]
    \item Supernova explosion
    \item Black hole
    \item White dwarf
    \item Neutron star
  \end{enumerate}

  \item Hubble's Law states that:
  \begin{enumerate}[label=(\Alph*)]
    \item Galaxies rotate at constant velocity
    \item The universe is contracting
    \item Galaxies recede at velocities proportional to their distance
    \item Light bends around massive objects
  \end{enumerate}

  \item The Cosmic Microwave Background radiation is evidence of:
  \begin{enumerate}[label=(\Alph*)]
    \item Stellar nucleosynthesis
    \item The early hot, dense universe
    \item Dark matter distribution
    \item Supernova remnants
  \end{enumerate}

  \item Red giant stars are larger but cooler at the surface than main-sequence stars of the same mass. (True/False)

  \item Black holes can be directly observed because they emit large amounts of light. (True/False)

  \item The observable universe contains approximately the same number of stars as grains of sand on Earth. (True/False)

  \item Describe the life cycle of a massive star (greater than 8 solar masses) from formation to its final state. What nuclear processes occur at each stage, and what determines the final outcome?
  \vspace{8em}

  \item Explain the evidence supporting the Big Bang theory. Discuss at least three key observations and how they support the model of an expanding universe that began from a hot, dense state.
  \vspace{8em}

\end{enumerate}

\end{document}