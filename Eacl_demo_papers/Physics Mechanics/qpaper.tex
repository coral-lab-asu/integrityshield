\documentclass[11pt]{article}

\usepackage[margin=1in]{geometry}
\usepackage{enumitem}
\usepackage{amsmath, amssymb}
\usepackage{hyperref}

\setlist[enumerate]{itemsep=0.8em, topsep=0.8em}
\setlist[itemize]{itemsep=0.3em, topsep=0.3em}

\begin{document}

\begin{center}
  {\Large \textbf{Classical Mechanics}}\\[0.25em]
  {\large \textbf{Newton's Laws and Energy Quiz}}\\[0.5em]
\end{center}

\noindent\textbf{Instructions:}
\begin{itemize}
  \item Answer all questions.
  \item For Questions 1--5, choose the best option.
  \item For Questions 6--8, mark True or False.
  \item For Questions 9--10, write detailed answers showing derivations where appropriate.
\end{itemize}

\vspace{0.5em}

\begin{enumerate}[label=\textbf{\arabic*.}]
  \item A 5 kg object accelerates at 2 m/s\textsuperscript{2}. What is the net force acting on it?
  \begin{enumerate}[label=(\Alph*)]
    \item 2.5 N
    \item 7 N
    \item 10 N
    \item 25 N
  \end{enumerate}

  \item An object moving in a circle at constant speed experiences:
  \begin{enumerate}[label=(\Alph*)]
    \item No acceleration
    \item Tangential acceleration only
    \item Centripetal acceleration toward the center
    \item Acceleration away from the center
  \end{enumerate}

  \item If the velocity of an object is doubled, its kinetic energy:
  \begin{enumerate}[label=(\Alph*)]
    \item Doubles
    \item Triples
    \item Quadruples
    \item Remains the same
  \end{enumerate}

  \item Which of the following is a statement of Newton's Third Law?
  \begin{enumerate}[label=(\Alph*)]
    \item $F = ma$
    \item Objects at rest stay at rest unless acted upon
    \item For every action, there is an equal and opposite reaction
    \item Energy cannot be created or destroyed
  \end{enumerate}

  \item A ball is thrown vertically upward. At the highest point, its:
  \begin{enumerate}[label=(\Alph*)]
    \item Velocity and acceleration are both zero
    \item Velocity is zero but acceleration is $g$ downward
    \item Velocity is maximum and acceleration is zero
    \item Velocity and acceleration are both $g$
  \end{enumerate}

  \item The work done by a force perpendicular to displacement is zero. (True/False)

  \item Momentum is conserved only in elastic collisions. (True/False)

  \item An object in free fall near Earth's surface experiences constant velocity. (True/False)

  \item Derive the work-energy theorem and explain its physical significance. Show how net work done on an object equals the change in its kinetic energy.
  \vspace{8em}

  \item Explain the principle of conservation of mechanical energy. Under what conditions is mechanical energy conserved? Discuss with an example of a pendulum or roller coaster.
  \vspace{8em}

\end{enumerate}

\end{document}