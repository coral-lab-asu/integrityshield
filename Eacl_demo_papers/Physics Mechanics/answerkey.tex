\documentclass[11pt]{article}

\usepackage[margin=1in]{geometry}
\usepackage{enumitem}
\usepackage{amsmath, amssymb}
\usepackage{hyperref}

\setlist[enumerate]{itemsep=0.8em, topsep=0.8em}

\begin{document}

\begin{center}
  {\Large \textbf{Classical Mechanics}}\\[0.25em]
  {\large \textbf{Newton's Laws and Energy Quiz -- Answer Key}}\\[0.5em]
\end{center}

\begin{enumerate}[label=\textbf{\arabic*.}]
  \item \textbf{(C) 10 N.} Using Newton's Second Law: $F = ma = 5 \text{ kg} \times 2 \text{ m/s}^2 = 10$ N.

  \item \textbf{(C) Centripetal acceleration toward the center.} Even at constant speed, direction changes continuously, requiring acceleration toward the center: $a_c = v^2/r$.

  \item \textbf{(C) Quadruples.} Kinetic energy $KE = \frac{1}{2}mv^2$. If $v \rightarrow 2v$, then $KE \rightarrow \frac{1}{2}m(2v)^2 = 4 \times \frac{1}{2}mv^2$.

  \item \textbf{(C) For every action, there is an equal and opposite reaction.} Newton's Third Law states forces always occur in pairs of equal magnitude and opposite direction acting on different objects.

  \item \textbf{(B) Velocity is zero but acceleration is $g$ downward.} At the apex, instantaneous velocity is zero, but gravitational acceleration ($g = 9.8$ m/s\textsuperscript{2}) acts continuously throughout the motion.

  \item \textbf{True.} Work $W = F \cdot d \cdot \cos\theta$. When $\theta = 90°$, $\cos 90° = 0$, so $W = 0$. Example: centripetal force does no work on circular motion.

  \item \textbf{False.} Momentum is conserved in all collisions (elastic and inelastic) when no external forces act. Kinetic energy is conserved only in elastic collisions.

  \item \textbf{False.} In free fall (neglecting air resistance), an object experiences constant acceleration ($g$), not constant velocity. Velocity continuously increases.

  \item \textbf{Derivation:}
  
  Starting with Newton's Second Law: $F = ma$
  
  For displacement $d$ in the direction of force:
  \begin{align*}
  W &= F \cdot d = ma \cdot d
  \end{align*}
  
  Using kinematic equation $v^2 = v_0^2 + 2ad$:
  \begin{align*}
  ad &= \frac{v^2 - v_0^2}{2}
  \end{align*}
  
  Substituting:
  \begin{align*}
  W &= m \cdot \frac{v^2 - v_0^2}{2} = \frac{1}{2}mv^2 - \frac{1}{2}mv_0^2 = \Delta KE
  \end{align*}
  
  \textbf{Work-Energy Theorem:} $W_{net} = \Delta KE = KE_f - KE_i$
  
  \textbf{Physical significance:}
  \begin{itemize}
    \item Net work done on an object equals change in kinetic energy
    \item Positive work increases kinetic energy (speeds up object)
    \item Negative work decreases kinetic energy (slows down object)
    \item Provides energy-based alternative to force analysis
    \item Useful when forces vary with position
  \end{itemize}

  \item \textbf{Principle:} In an isolated system with only conservative forces, the total mechanical energy (kinetic + potential) remains constant:
  \begin{align*}
  E_{total} = KE + PE = \text{constant}
  \end{align*}
  
  \textbf{Conditions for conservation:}
  \begin{itemize}
    \item Only conservative forces do work (gravity, spring force)
    \item No non-conservative forces (friction, air resistance, applied forces)
    \item System is isolated (no external work)
  \end{itemize}
  
  \textbf{Pendulum example:}
  \begin{itemize}
    \item At maximum height: $KE = 0$, $PE = mgh_{max}$ (all potential)
    \item At lowest point: $KE = \frac{1}{2}mv_{max}^2$, $PE = 0$ (all kinetic)
    \item Energy transforms between KE and PE: $mgh = \frac{1}{2}mv^2$
    \item Maximum speed: $v_{max} = \sqrt{2gh}$
  \end{itemize}
  
  \textbf{Roller coaster example:}
  \begin{itemize}
    \item At top of first hill (height $h_1$): $E = mgh_1$ (starting from rest)
    \item At any other point (height $h_2$): $mgh_1 = \frac{1}{2}mv^2 + mgh_2$
    \item Speed at height $h_2$: $v = \sqrt{2g(h_1 - h_2)}$
    \item Coaster cannot rise higher than starting height without external work
  \end{itemize}

\end{enumerate}

\end{document}