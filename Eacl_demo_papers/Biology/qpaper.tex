\documentclass[11pt]{article}

\usepackage[margin=1in]{geometry}
\usepackage{enumitem}
\usepackage{amsmath, amssymb}
\usepackage{hyperref}

\setlist[enumerate]{itemsep=0.8em, topsep=0.8em}
\setlist[itemize]{itemsep=0.3em, topsep=0.3em}

\begin{document}

\begin{center}
  {\Large \textbf{Introduction to Biology}}\\[0.25em]
  {\large \textbf{Cell Biology and Genetics Quiz}}\\[0.5em]
\end{center}

\noindent\textbf{Instructions:}
\begin{itemize}
  \item Answer all questions.
  \item For Questions 1--5, choose the best option.
  \item For Questions 6--8, mark True or False.
  \item For Questions 9--10, write detailed answers with biological explanations.
\end{itemize}

\vspace{0.5em}

\begin{enumerate}[label=\textbf{\arabic*.}]
  \item Which organelle is responsible for ATP production in eukaryotic cells?
  \begin{enumerate}[label=(\Alph*)]
    \item Ribosome
    \item Golgi apparatus
    \item Mitochondrion
    \item Endoplasmic reticulum
  \end{enumerate}

  \item The process by which mRNA is synthesized from a DNA template is called:
  \begin{enumerate}[label=(\Alph*)]
    \item Translation
    \item Replication
    \item Transcription
    \item Transduction
  \end{enumerate}

  \item In Mendelian genetics, if a heterozygous tall plant (Tt) is crossed with a homozygous short plant (tt), what percentage of offspring will be tall?
  \begin{enumerate}[label=(\Alph*)]
    \item 25\%
    \item 50\%
    \item 75\%
    \item 100\%
  \end{enumerate}

  \item Which of the following is NOT a function of the cell membrane?
  \begin{enumerate}[label=(\Alph*)]
    \item Selective permeability
    \item Cell signaling
    \item Protein synthesis
    \item Cell adhesion
  \end{enumerate}

  \item Sister chromatids separate during which phase of cell division?
  \begin{enumerate}[label=(\Alph*)]
    \item Prophase
    \item Metaphase
    \item Anaphase
    \item Telophase
  \end{enumerate}

  \item All cells contain membrane-bound organelles such as mitochondria and nuclei. (True/False)

  \item DNA replication is described as ``semi-conservative'' because each new double helix contains one original and one newly synthesized strand. (True/False)

  \item Mutations are always harmful to an organism. (True/False)

  \item Describe the structure of DNA and explain how its structure enables its function as the molecule of heredity. Include the concepts of complementary base pairing and the double helix.
  \vspace{8em}

  \item Compare and contrast mitosis and meiosis. Discuss the purposes of each process and explain why meiosis produces genetic variation while mitosis does not.
  \vspace{8em}

\end{enumerate}

\end{document}