\documentclass[11pt]{article}

\usepackage[margin=1in]{geometry}
\usepackage{enumitem}
\usepackage{amsmath, amssymb}
\usepackage{hyperref}

\setlist[enumerate]{itemsep=0.8em, topsep=0.8em}

\begin{document}

\begin{center}
  {\Large \textbf{Introduction to Biology}}\\[0.25em]
  {\large \textbf{Cell Biology and Genetics Quiz -- Answer Key}}\\[0.5em]
\end{center}

\begin{enumerate}[label=\textbf{\arabic*.}]
  \item \textbf{(C) Mitochondrion.} Mitochondria are the ``powerhouses'' of the cell, generating ATP through oxidative phosphorylation in cellular respiration.

  \item \textbf{(C) Transcription.} Transcription occurs in the nucleus where RNA polymerase reads DNA and synthesizes complementary mRNA. Translation (protein synthesis) occurs at ribosomes.

  \item \textbf{(B) 50\%.} Punnett square: Tt $\times$ tt yields Tt, Tt, tt, tt offspring—50\% heterozygous tall (Tt), 50\% homozygous short (tt).

  \item \textbf{(C) Protein synthesis.} Protein synthesis occurs at ribosomes (in cytoplasm and on rough ER). The cell membrane regulates transport, signaling, and adhesion.

  \item \textbf{(C) Anaphase.} During anaphase, the centromeres split and sister chromatids are pulled to opposite poles by spindle fibers.

  \item \textbf{False.} Prokaryotes (bacteria and archaea) lack membrane-bound organelles. They have no nucleus, mitochondria, or other membrane-enclosed structures—only ribosomes and a nucleoid region.

  \item \textbf{True.} Meselson and Stahl demonstrated semi-conservative replication. Each daughter DNA molecule retains one parental strand paired with one newly synthesized strand.

  \item \textbf{False.} Mutations can be harmful, neutral, or beneficial. Beneficial mutations provide raw material for evolution through natural selection. Many mutations have no phenotypic effect.

  \item \textbf{Structure:} DNA is a double helix of two antiparallel polynucleotide strands. Each strand has a sugar-phosphate backbone with nitrogenous bases projecting inward. The four bases are adenine (A), thymine (T), guanine (G), and cytosine (C).
  
  \textbf{Complementary base pairing:} Hydrogen bonds connect bases between strands—A pairs with T (2 H-bonds), G pairs with C (3 H-bonds). This specificity means each strand contains information to reconstruct the other.
  
  \textbf{Structure-function relationships:}
  \begin{itemize}
    \item \textit{Information storage:} The sequence of bases encodes genetic information in codons (triplets specifying amino acids)
    \item \textit{Replication:} Complementarity allows accurate copying—each strand templates synthesis of its partner
    \item \textit{Stability:} Double helix and H-bonds protect genetic information; sugar-phosphate backbone resists hydrolysis
    \item \textit{Accessibility:} Helix can unwind for transcription and replication
  \end{itemize}

  \item \textbf{Mitosis:}
  \begin{itemize}
    \item One division producing two diploid (2n) daughter cells
    \item Daughter cells are genetically identical to parent
    \item Purpose: growth, repair, asexual reproduction
    \item Occurs in somatic (body) cells
  \end{itemize}
  
  \textbf{Meiosis:}
  \begin{itemize}
    \item Two divisions producing four haploid (n) daughter cells
    \item Daughter cells are genetically unique
    \item Purpose: produce gametes for sexual reproduction
    \item Occurs in germ cells (gonads)
  \end{itemize}
  
  \textbf{Sources of genetic variation in meiosis:}
  
  (1) \textit{Crossing over:} During prophase I, homologous chromosomes exchange segments, creating new allele combinations.
  
  (2) \textit{Independent assortment:} Homologous pairs orient randomly at metaphase I; with 23 pairs, $2^{23}$ ($>$8 million) combinations are possible.
  
  (3) \textit{Random fertilization:} Any sperm can fertilize any egg, multiplying variation.
  
  \textbf{Why mitosis lacks variation:} No homologous pairing, no crossing over, sister chromatids (identical copies) separate rather than homologs.

\end{enumerate}

\end{document}