\documentclass[11pt]{article}

\usepackage[margin=1in]{geometry}
\usepackage{enumitem}
\usepackage{amsmath, amssymb}
\usepackage{hyperref}

\setlist[enumerate]{itemsep=0.8em, topsep=0.8em}

\begin{document}

\begin{center}
  {\Large \textbf{Electrical Circuits}}\\[0.25em]
  {\large \textbf{DC Circuit Analysis Quiz -- Answer Key}}\\[0.5em]
\end{center}

\begin{enumerate}[label=\textbf{\arabic*.}]
  \item \textbf{(C) 2$\Omega$.} For parallel resistors: $1/R_{eq} = 1/6 + 1/6 + 1/6 = 3/6 = 1/2$, so $R_{eq} = 2\Omega$.

  \item \textbf{(B) The sum of currents entering a node equals the sum leaving.} KCL is based on conservation of charge—no charge accumulates at a node: $\sum I_{in} = \sum I_{out}$.

  \item \textbf{(B) 36W.} Current $I = V/R = 12/4 = 3$A. Power $P = I^2R = 9 \times 4 = 36$W. Or $P = V^2/R = 144/4 = 36$W.

  \item \textbf{(B) Current.} In series, components share the same current path. Voltage divides across components; total resistance is the sum of individual resistances.

  \item \textbf{(B) $R \times C$.} Time constant $\tau = RC$ (in seconds when R in ohms, C in farads). After one time constant, capacitor charges to ~63.2\% of final value.

  \item \textbf{True.} An ideal voltage source has zero internal resistance and maintains constant terminal voltage regardless of load current (infinite current capability in theory).

  \item \textbf{False.} Thevenin's theorem states a linear circuit can be replaced by a voltage source in series with a resistor ($V_{Th}$ in series with $R_{Th}$). Norton's theorem uses a current source in parallel with a resistor.

  \item \textbf{True.} Maximum power transfer theorem: $P_{max}$ occurs when $R_{load} = R_{source}$. At this point, efficiency is 50\% (half the power dissipated in source resistance).

  \item \textbf{Kirchhoff's Voltage Law (KVL):}
  \begin{itemize}
    \item Statement: The algebraic sum of voltages around any closed loop equals zero
    \item Based on conservation of energy
    \item Sign convention: Voltage rises positive, drops negative (or vice versa, consistently)
    \item Equation: $\sum V = 0$ around any closed path
  \end{itemize}
  
  \textbf{Kirchhoff's Current Law (KCL):}
  \begin{itemize}
    \item Statement: Sum of currents entering a node equals sum leaving
    \item Based on conservation of charge
    \item Equation: $\sum I_{in} = \sum I_{out}$ or $\sum I = 0$ (with sign convention)
  \end{itemize}
  
  \textbf{Analysis procedure:}
  \begin{enumerate}
    \item Identify nodes and assign node voltages or loop currents
    \item Apply KCL at each node (for nodal analysis)
    \item Apply KVL around each independent loop (for mesh analysis)
    \item Solve simultaneous equations
  \end{enumerate}
  
  \textbf{Example (two-loop circuit):}
  Loop 1: $V_s - I_1R_1 - (I_1-I_2)R_2 = 0$
  Loop 2: $(I_2-I_1)R_2 - I_2R_3 = 0$

  \item \textbf{Thevenin's Theorem:}
  \begin{itemize}
    \item Any linear two-terminal circuit can be replaced by a voltage source $V_{Th}$ in series with a resistance $R_{Th}$
    \item $V_{Th}$ = open-circuit voltage across terminals
    \item $R_{Th}$ = equivalent resistance seen from terminals with sources deactivated (voltage sources shorted, current sources opened)
  \end{itemize}
  
  \textbf{Norton's Theorem:}
  \begin{itemize}
    \item Any linear two-terminal circuit can be replaced by a current source $I_N$ in parallel with a resistance $R_N$
    \item $I_N$ = short-circuit current between terminals
    \item $R_N$ = equivalent resistance (same as $R_{Th}$)
  \end{itemize}
  
  \textbf{Finding Thevenin equivalent:}
  \begin{enumerate}
    \item Remove the load from the circuit
    \item Calculate $V_{Th}$: voltage across open terminals
    \item Calculate $R_{Th}$: deactivate sources, find equivalent resistance
    \item Reconnect load to Thevenin equivalent
  \end{enumerate}
  
  \textbf{Relationship between Thevenin and Norton:}
  \begin{align*}
  V_{Th} &= I_N \times R_N \\
  R_{Th} &= R_N \\
  I_N &= V_{Th} / R_{Th}
  \end{align*}
  
  Thevenin and Norton are source transformations—either can represent the same circuit behavior at the terminals.

\end{enumerate}

\end{document}