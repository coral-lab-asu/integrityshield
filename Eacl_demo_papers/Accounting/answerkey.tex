\documentclass[11pt]{article}

\usepackage[margin=1in]{geometry}
\usepackage{enumitem}
\usepackage{amsmath, amssymb}
\usepackage{hyperref}

\setlist[enumerate]{itemsep=0.8em, topsep=0.8em}

\begin{document}

\begin{center}
  {\Large \textbf{Financial Accounting}}\\[0.25em]
  {\large \textbf{Financial Statements and Analysis Quiz -- Answer Key}}\\[0.5em]
\end{center}

\begin{enumerate}[label=\textbf{\arabic*.}]
  \item \textbf{(B) Assets = Liabilities + Equity.} This fundamental equation must always balance. Assets are what the company owns; liabilities and equity represent how assets are financed.

  \item \textbf{(B) Balance Sheet.} The balance sheet shows assets, liabilities, and equity at a specific date. Income statement and cash flow statement cover periods of time.

  \item \textbf{(B) Allocate the cost of an asset over its useful life.} Depreciation matches the expense of using an asset to the revenues it generates (matching principle). It's a non-cash expense.

  \item \textbf{(B) Liquidity.} Current ratio = Current Assets / Current Liabilities. It measures ability to pay short-term obligations. Values above 1 indicate sufficient short-term resources.

  \item \textbf{(C) The earnings process is substantially complete and collection is reasonably assured.} Revenue recognition under accrual accounting follows the realization principle, not cash receipt timing.

  \item \textbf{False.} An increase in accounts receivable means sales were made but cash wasn't collected—it's a use of cash (subtracted in operating activities under indirect method).

  \item \textbf{True.} LIFO assigns newer (higher) costs to COGS during inflation, resulting in higher expenses and lower net income compared to FIFO, which assigns older (lower) costs.

  \item \textbf{True.} Goodwill = Purchase Price - Fair Value of Net Identifiable Assets. It represents intangibles like brand value, customer relationships, and synergies.

  \item \textbf{Income Statement:}
  \begin{itemize}
    \item Reports financial performance over a period
    \item Shows revenues, expenses, and net income
    \item Answers: ``How profitable was the company?''
    \item Key items: Revenue, COGS, Operating Expenses, Net Income
  \end{itemize}
  
  \textbf{Balance Sheet:}
  \begin{itemize}
    \item Reports financial position at a point in time
    \item Shows assets, liabilities, and equity
    \item Answers: ``What does the company own and owe?''
    \item Must balance: Assets = Liabilities + Equity
  \end{itemize}
  
  \textbf{Cash Flow Statement:}
  \begin{itemize}
    \item Reports cash movements over a period
    \item Three sections: Operating, Investing, Financing activities
    \item Answers: ``Where did cash come from and go?''
    \item Reconciles net income to cash changes
  \end{itemize}
  
  \textbf{Interconnections:}
  \begin{itemize}
    \item Net income (Income Statement) → flows to Retained Earnings (Balance Sheet)
    \item Net income → starting point for operating cash flow (Cash Flow Statement)
    \item Ending cash (Cash Flow Statement) → Cash asset (Balance Sheet)
    \item Retained Earnings connects all three statements
  \end{itemize}

  \item \textbf{Cash Basis Accounting:}
  \begin{itemize}
    \item Records revenue when cash received
    \item Records expenses when cash paid
    \item Simple but doesn't match economic activity
    \item Example: December service, January payment → revenue in January
  \end{itemize}
  
  \textbf{Accrual Basis Accounting:}
  \begin{itemize}
    \item Records revenue when earned (regardless of cash receipt)
    \item Records expenses when incurred (regardless of payment)
    \item Matches revenues with related expenses
    \item Example: December service, January payment → revenue in December
  \end{itemize}
  
  \textbf{Why GAAP/IFRS require accrual:}
  \begin{itemize}
    \item Better matches economic reality
    \item Enables meaningful period-to-period comparisons
    \item Prevents manipulation through payment timing
    \item Provides more relevant information for decision-making
    \item Supports the matching principle
  \end{itemize}
  
  \textbf{Examples:}
  \begin{itemize}
    \item \textit{Accounts Receivable:} Under accrual, credit sales recorded as revenue immediately; cash basis waits for payment
    \item \textit{Prepaid Insurance:} Under accrual, expensed monthly as coverage used; cash basis records full expense when paid
    \item \textit{Accrued Wages:} Under accrual, expense recorded as employees work; cash basis records only when paid
  \end{itemize}

\end{enumerate}

\end{document}