\documentclass[11pt]{article}

\usepackage[margin=1in]{geometry}
\usepackage{enumitem}
\usepackage{amsmath, amssymb}
\usepackage{hyperref}

\setlist[enumerate]{itemsep=0.8em, topsep=0.8em}

\begin{document}

\begin{center}
  {\Large \textbf{Introduction to Political Science}}\\[0.25em]
  {\large \textbf{Democratic Theory and Political Systems Quiz -- Answer Key}}\\[0.5em]
\end{center}

\begin{enumerate}[label=\textbf{\arabic*.}]
  \item \textbf{(C) Thomas Hobbes.} In \textit{Leviathan} (1651), Hobbes argued individuals surrender freedoms to a sovereign authority in exchange for security—the social contract.

  \item \textbf{(B) The legislature.} In parliamentary systems (UK, Canada, Germany), the prime minister and cabinet are drawn from and accountable to parliament. They govern with legislative confidence.

  \item \textbf{(B) Montesquieu.} In \textit{The Spirit of the Laws} (1748), Montesquieu argued liberty is protected by separating legislative, executive, and judicial powers.

  \item \textbf{(B) Distribution of power among multiple competing interest groups.} Pluralists (Robert Dahl) argue democracy functions through bargaining among diverse organized interests rather than unified elite rule.

  \item \textbf{(C) First-past-the-post (plurality).} Duverger's Law: single-member plurality systems tend toward two parties because votes for third parties are ``wasted,'' encouraging strategic voting.

  \item \textbf{True.} Federalism (US, Germany, India) constitutionally divides authority between levels of government, each with defined powers. This contrasts with unitary systems where central government holds sovereignty.

  \item \textbf{False.} Many authoritarian regimes hold elections (Russia, Iran, Singapore) but manipulate them through restrictions on opposition, media control, or fraud. These are ``competitive authoritarian'' or ``hybrid'' regimes.

  \item \textbf{True.} Mill and Tocqueville warned that unchecked majorities could violate individual rights. Constitutional democracies include counter-majoritarian protections (rights, judicial review).

  \item \textbf{Presidential System (USA, Brazil):}
  \begin{itemize}
    \item Executive elected separately from legislature
    \item Fixed terms; president cannot be removed by legislative vote (except impeachment)
    \item Strict separation of powers
  \end{itemize}
  
  \textit{Advantages:} Stability, clear accountability, checks and balances prevent concentration of power.
  
  \textit{Disadvantages:} Gridlock when executive and legislature disagree; difficult to remove incompetent leaders; winner-take-all can exclude minorities.
  
  \textbf{Parliamentary System (UK, Germany, India):}
  \begin{itemize}
    \item Executive (PM/cabinet) drawn from legislature
    \item Government requires parliamentary confidence; can be removed by vote of no confidence
    \item Fusion of powers
  \end{itemize}
  
  \textit{Advantages:} Flexibility to change leadership; unified government enables decisive action; coalition-building includes diverse voices.
  
  \textit{Disadvantages:} Less stable governments (frequent elections possible); fewer checks on majority; weaker separation of powers.
  
  \textbf{Conclusion:} Neither is inherently superior; effectiveness depends on political culture, party systems, and institutional design.

  \item \textbf{Essential features of liberal democracy:}
  \begin{itemize}
    \item Free, fair, competitive elections with universal suffrage
    \item Rule of law—government bound by legal constraints
    \item Protection of individual rights and civil liberties
    \item Separation of powers and institutional checks
    \item Independent judiciary
    \item Free press and civil society
  \end{itemize}
  
  \textbf{Tension—majority rule vs. minority rights:} Democracy means popular sovereignty, but unchecked majorities can oppress minorities. Historical examples: segregation, persecution of religious minorities, discrimination against LGBTQ+ individuals.
  
  \textbf{Protective mechanisms:}
  
  (1) \textit{Constitutional rights:} Enumerate protections (speech, religion, due process) beyond majority repeal.
  
  (2) \textit{Judicial review:} Courts invalidate laws violating constitutional rights (Marbury v. Madison).
  
  (3) \textit{Supermajority requirements:} Constitutional amendments require more than simple majorities.
  
  (4) \textit{Federal structures:} Dispersing power protects local/regional minorities.
  
  (5) \textit{Electoral design:} Proportional representation ensures minority voices in legislature.
  
  \textbf{Conclusion:} Liberal democracy requires balancing majoritarianism with constraints protecting individual dignity and minority rights.

\end{enumerate}

\end{document}