\documentclass[11pt]{article}

\usepackage[margin=1in]{geometry}
\usepackage{enumitem}
\usepackage{amsmath, amssymb}
\usepackage{hyperref}

\setlist[enumerate]{itemsep=0.8em, topsep=0.8em}
\setlist[itemize]{itemsep=0.3em, topsep=0.3em}

\begin{document}

\begin{center}
  {\Large \textbf{Introduction to Political Science}}\\[0.25em]
  {\large \textbf{Democratic Theory and Political Systems Quiz}}\\[0.5em]
\end{center}

\noindent\textbf{Instructions:}
\begin{itemize}
  \item Answer all questions.
  \item For Questions 1--5, choose the best option.
  \item For Questions 6--8, mark True or False.
  \item For Questions 9--10, write detailed answers with analytical arguments.
\end{itemize}

\vspace{0.5em}

\begin{enumerate}[label=\textbf{\arabic*.}]
  \item Which political philosopher articulated the concept of the ``social contract'' in \textit{Leviathan}?
  \begin{enumerate}[label=(\Alph*)]
    \item John Locke
    \item Jean-Jacques Rousseau
    \item Thomas Hobbes
    \item Montesquieu
  \end{enumerate}

  \item In a parliamentary system, the executive branch derives its legitimacy from:
  \begin{enumerate}[label=(\Alph*)]
    \item Direct popular election
    \item The legislature
    \item The judiciary
    \item The military
  \end{enumerate}

  \item The principle of ``separation of powers'' was most influentially articulated by:
  \begin{enumerate}[label=(\Alph*)]
    \item Niccolò Machiavelli
    \item Montesquieu
    \item John Stuart Mill
    \item Edmund Burke
  \end{enumerate}

  \item ``Pluralism'' in democratic theory refers to:
  \begin{enumerate}[label=(\Alph*)]
    \item Rule by a single dominant party
    \item Distribution of power among multiple competing interest groups
    \item Government by technical experts
    \item Direct citizen participation in all decisions
  \end{enumerate}

  \item Which electoral system is most likely to produce a two-party system?
  \begin{enumerate}[label=(\Alph*)]
    \item Proportional representation
    \item Single transferable vote
    \item First-past-the-post (plurality)
    \item Mixed-member proportional
  \end{enumerate}

  \item In a federal system, sovereignty is divided between national and subnational governments. (True/False)

  \item Authoritarian regimes never hold elections. (True/False)

  \item The ``tyranny of the majority'' refers to the danger that democratic majorities may oppress minority rights. (True/False)

  \item Compare presidential and parliamentary systems of government. Discuss the advantages and disadvantages of each, using specific country examples.
  \vspace{8em}

  \item What are the essential features of liberal democracy? Discuss the tensions between majority rule and minority rights, and explain the institutional mechanisms designed to protect individual liberties.
  \vspace{8em}

\end{enumerate}

\end{document}