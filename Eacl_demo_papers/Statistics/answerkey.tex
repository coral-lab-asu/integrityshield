\documentclass[11pt]{article}

\usepackage[margin=1in]{geometry}
\usepackage{enumitem}
\usepackage{amsmath, amssymb}
\usepackage{hyperref}

\setlist[enumerate]{itemsep=0.8em, topsep=0.8em}

\begin{document}

\begin{center}
  {\Large \textbf{Introduction to Statistics}}\\[0.25em]
  {\large \textbf{Probability and Inference Quiz -- Answer Key}}\\[0.5em]
\end{center}

\begin{enumerate}[label=\textbf{\arabic*.}]
  \item \textbf{(B) 0.12.} For independent events, P(A $\cap$ B) = P(A) $\times$ P(B) = 0.3 $\times$ 0.4 = 0.12.

  \item \textbf{(B) Sample means approach a normal distribution as sample size increases.} Regardless of the population distribution, the sampling distribution of the mean becomes approximately normal for large n (typically n $\geq$ 30).

  \item \textbf{(B) Rejecting a true null hypothesis.} Type I error (false positive) occurs when we incorrectly conclude there is an effect. Type II error is failing to reject a false null hypothesis.

  \item \textbf{(B) Median.} The median is the middle value and is not affected by extreme values. The mean is pulled toward outliers, and range is directly determined by extremes.

  \item \textbf{(C) 95\% of similarly constructed intervals would contain the true parameter.} This is the frequentist interpretation—the procedure captures the parameter 95\% of the time, not probability statements about any single interval.

  \item \textbf{True.} The standard error (SE) is the standard deviation of a sampling distribution. For sample means: SE = $\sigma/\sqrt{n}$.

  \item \textbf{False.} The p-value is the probability of observing data this extreme or more extreme, assuming the null hypothesis is true. It is not the probability that the null hypothesis is true.

  \item \textbf{False.} Correlation measures linear association but does not establish causation. Causation requires controlled experiments, temporal precedence, and elimination of confounding variables.

  \item \textbf{Descriptive Statistics:}
  \begin{itemize}
    \item Summarizes and describes data characteristics
    \item Measures: mean, median, mode, standard deviation, range, percentiles
    \item Visualizations: histograms, box plots, scatter plots
    \item Example: Calculating average test scores for a class, creating a frequency distribution of survey responses
  \end{itemize}
  
  \textbf{Inferential Statistics:}
  \begin{itemize}
    \item Makes generalizations about populations from samples
    \item Techniques: hypothesis testing, confidence intervals, regression
    \item Accounts for sampling variability and uncertainty
    \item Example: Using a sample of 500 voters to estimate support for a candidate in the entire population
  \end{itemize}
  
  \textbf{Role of sampling:}
  \begin{itemize}
    \item Random sampling ensures representativeness
    \item Larger samples reduce sampling error
    \item Sampling distributions allow probability calculations
    \item Enables generalization from sample statistics to population parameters
  \end{itemize}

  \item \textbf{Hypothesis Testing Steps:}
  
  \textbf{(1) State hypotheses:}
  \begin{itemize}
    \item Null hypothesis (H$_0$): No effect or no difference (status quo)
    \item Alternative hypothesis (H$_a$): Effect exists or difference present
    \item Example: H$_0$: $\mu$ = 100 (mean IQ unchanged); H$_a$: $\mu$ $\neq$ 100 (mean IQ differs)
  \end{itemize}
  
  \textbf{(2) Set significance level ($\alpha$):}
  \begin{itemize}
    \item Typically $\alpha$ = 0.05 (5\% chance of Type I error)
    \item Determines critical region for rejection
  \end{itemize}
  
  \textbf{(3) Collect data and calculate test statistic:}
  \begin{itemize}
    \item Example: Sample of n=36 students, $\bar{x}$ = 105, s = 15
    \item t = ($\bar{x}$ - $\mu_0$)/(s/$\sqrt{n}$) = (105 - 100)/(15/$\sqrt{36}$) = 5/2.5 = 2.0
  \end{itemize}
  
  \textbf{(4) Calculate p-value:}
  \begin{itemize}
    \item Probability of observing test statistic this extreme under H$_0$
    \item For t = 2.0 with df = 35, two-tailed p $\approx$ 0.053
  \end{itemize}
  
  \textbf{(5) Make decision:}
  \begin{itemize}
    \item If p $\leq$ $\alpha$: Reject H$_0$ (statistically significant)
    \item If p > $\alpha$: Fail to reject H$_0$ (not statistically significant)
    \item Example: p = 0.053 > 0.05, so fail to reject H$_0$
  \end{itemize}
  
  \textbf{(6) State conclusion in context:}
  \begin{itemize}
    \item Example: ``There is insufficient evidence at the 0.05 level to conclude that mean IQ differs from 100.''
  \end{itemize}

\end{enumerate}

\end{document}