\documentclass[11pt]{article}

\usepackage[margin=1in]{geometry}
\usepackage{enumitem}
\usepackage{amsmath, amssymb}
\usepackage{hyperref}

\setlist[enumerate]{itemsep=0.8em, topsep=0.8em}

\begin{document}

\begin{center}
  {\Large \textbf{Computer Networks}}\\[0.25em]
  {\large \textbf{TCP/IP and Network Architecture Quiz -- Answer Key}}\\[0.5em]
\end{center}

\begin{enumerate}[label=\textbf{\arabic*.}]
  \item \textbf{(B) Network Layer.} Layer 3 (Network) handles logical addressing (IP) and routing packets between different networks. Data Link handles local delivery; Transport handles end-to-end communication.

  \item \textbf{(B) Acknowledgments and retransmissions.} TCP uses sequence numbers, acknowledgments, checksums, and retransmission timers to ensure reliable, ordered delivery of data.

  \item \textbf{(B) Translating domain names to IP addresses.} DNS is a hierarchical naming system that resolves human-readable domain names (e.g., google.com) to IP addresses (e.g., 142.250.80.46).

  \item \textbf{(B) UDP.} User Datagram Protocol is connectionless and unreliable (no acknowledgments or retransmissions). TCP is connection-oriented and reliable; IP is Network Layer; HTTP is Application Layer.

  \item \textbf{(B) 254.} A /24 network has 256 addresses (2\textsuperscript{8}), but the network address (first) and broadcast address (last) are reserved, leaving 254 usable host addresses.

  \item \textbf{True.} ARP operates at the Data Link Layer to discover the MAC (hardware) address associated with an IP address on the local network segment.

  \item \textbf{False.} HTTP/2 specification allows unencrypted connections (h2c), though most browser implementations require TLS (h2). HTTP/3 (QUIC) similarly doesn't mandate but typically uses encryption.

  \item \textbf{False.} Switches operate at the Data Link Layer (Layer 2) and make forwarding decisions based on MAC addresses. Routers operate at the Network Layer and use IP addresses.

  \item \textbf{TCP Characteristics:}
  \begin{itemize}
    \item Connection-oriented (three-way handshake)
    \item Reliable delivery (acknowledgments, retransmissions)
    \item Ordered delivery (sequence numbers)
    \item Flow control (sliding window)
    \item Congestion control (slow start, congestion avoidance)
    \item Higher overhead
  \end{itemize}
  
  \textbf{UDP Characteristics:}
  \begin{itemize}
    \item Connectionless (no handshake)
    \item Unreliable (no acknowledgments or retransmissions)
    \item No ordering guarantees
    \item No flow/congestion control
    \item Lower overhead, faster
  \end{itemize}
  
  \textbf{TCP Use Cases:} Web browsing (HTTP), email (SMTP), file transfer (FTP), SSH—applications requiring complete, ordered data delivery.
  
  \textbf{UDP Use Cases:} Video streaming, VoIP, online gaming, DNS queries—applications tolerating some loss but requiring low latency; real-time applications where retransmission is counterproductive.
  
  \textbf{Selection criteria:} Choose TCP when data integrity is critical; choose UDP when speed/latency matters more than perfect delivery.

  \item \textbf{TCP Three-Way Handshake:}
  
  \textbf{Step 1 - SYN:}
  \begin{itemize}
    \item Client sends SYN (synchronize) segment to server
    \item Contains client's initial sequence number (ISN)
    \item Client enters SYN\_SENT state
  \end{itemize}
  
  \textbf{Step 2 - SYN-ACK:}
  \begin{itemize}
    \item Server receives SYN, responds with SYN-ACK
    \item Acknowledges client's ISN (ACK = client\_ISN + 1)
    \item Includes server's own ISN
    \item Server enters SYN\_RECEIVED state
  \end{itemize}
  
  \textbf{Step 3 - ACK:}
  \begin{itemize}
    \item Client acknowledges server's ISN (ACK = server\_ISN + 1)
    \item Client enters ESTABLISHED state
    \item Upon receipt, server enters ESTABLISHED state
    \item Connection ready for data transfer
  \end{itemize}
  
  \textbf{Purpose:}
  \begin{itemize}
    \item Establishes connection parameters
    \item Synchronizes sequence numbers for reliable transfer
    \item Confirms both parties are ready to communicate
  \end{itemize}
  
  \textbf{Packet loss handling:}
  \begin{itemize}
    \item If SYN lost: Client retransmits after timeout
    \item If SYN-ACK lost: Server retransmits; client may retransmit SYN
    \item If final ACK lost: Server retransmits SYN-ACK; client responds with ACK again
    \item Exponential backoff prevents network congestion
  \end{itemize}

\end{enumerate}

\end{document}