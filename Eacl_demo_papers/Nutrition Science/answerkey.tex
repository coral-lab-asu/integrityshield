\documentclass[11pt]{article}

\usepackage[margin=1in]{geometry}
\usepackage{enumitem}
\usepackage{amsmath, amssymb}
\usepackage{hyperref}

\setlist[enumerate]{itemsep=0.8em, topsep=0.8em}

\begin{document}

\begin{center}
  {\Large \textbf{Nutrition Science}}\\[0.25em]
  {\large \textbf{Macronutrients and Metabolism Quiz -- Answer Key}}\\[0.5em]
\end{center}

\begin{enumerate}[label=\textbf{\arabic*.}]
  \item \textbf{(C) Fats (9 kcal/g).} Fats are the most energy-dense macronutrient, providing more than twice the energy of carbohydrates or proteins per gram.

  \item \textbf{(B) How quickly a food raises blood glucose levels.} GI compares foods to pure glucose (GI=100). High-GI foods cause rapid spikes; low-GI foods cause gradual increases.

  \item \textbf{(B) Cannot be synthesized by the body and must be obtained from diet.} Nine essential amino acids (histidine, isoleucine, leucine, lysine, methionine, phenylalanine, threonine, tryptophan, valine) must come from food.

  \item \textbf{(C) Trans fatty acids.} Trans fats raise LDL cholesterol, lower HDL cholesterol, and increase inflammation—all risk factors for cardiovascular disease. Largely banned in food production.

  \item \textbf{(C) Coenzymes in metabolic reactions.} B vitamins (thiamin, riboflavin, niacin, B6, B12, folate, etc.) serve as coenzymes for energy metabolism, DNA synthesis, and other reactions.

  \item \textbf{False.} Fiber is not digested by human enzymes. Soluble fiber is fermented by gut bacteria in the colon; insoluble fiber passes through largely intact, adding bulk to stool.

  \item \textbf{True.} Complete proteins (eggs, meat, fish, dairy, soy, quinoa) provide all nine essential amino acids. Most plant proteins are incomplete (lacking or low in one or more).

  \item \textbf{True.} BMR accounts for 60-75\% of total daily energy expenditure in sedentary individuals, covering vital functions: breathing, circulation, cell production, brain function.

  \item \textbf{Glucose Metabolism Overview:}
  
  \textbf{Glycolysis (cytoplasm):}
  \begin{itemize}
    \item Glucose (6C) → 2 Pyruvate (3C)
    \item Net yield: 2 ATP + 2 NADH
    \item Does not require oxygen (anaerobic possible)
    \item 10-step enzyme-catalyzed pathway
  \end{itemize}
  
  \textbf{Pyruvate Processing (mitochondrial matrix):}
  \begin{itemize}
    \item Pyruvate → Acetyl-CoA (2C) + CO$_2$
    \item Produces 1 NADH per pyruvate
    \item Requires oxygen (aerobic)
  \end{itemize}
  
  \textbf{Citric Acid Cycle/Krebs Cycle (mitochondrial matrix):}
  \begin{itemize}
    \item Acetyl-CoA (2C) combines with oxaloacetate (4C)
    \item 8-step cycle regenerates oxaloacetate
    \item Per Acetyl-CoA: 3 NADH + 1 FADH$_2$ + 1 GTP + 2 CO$_2$
    \item Completes oxidation of glucose carbons
  \end{itemize}
  
  \textbf{Oxidative Phosphorylation (inner mitochondrial membrane):}
  \begin{itemize}
    \item Electron transport chain accepts electrons from NADH, FADH$_2$
    \item Creates proton gradient across membrane
    \item ATP synthase uses gradient to produce ATP
    \item Oxygen is final electron acceptor (forms H$_2$O)
    \item Produces ~26-28 ATP per glucose
  \end{itemize}
  
  \textbf{Total yield: ~30-32 ATP per glucose molecule}

  \item \textbf{Saturated Fats:}
  \begin{itemize}
    \item Structure: No double bonds; carbon chain fully ``saturated'' with hydrogens
    \item Solid at room temperature (straight chains pack tightly)
    \item Sources: Animal fats (butter, lard), coconut oil, palm oil
    \item Health effects: Raise LDL cholesterol; associated with increased cardiovascular risk
  \end{itemize}
  
  \textbf{Unsaturated Fats:}
  \begin{itemize}
    \item Structure: One (mono-) or more (poly-) double bonds; creates ``kinks''
    \item Liquid at room temperature (kinks prevent tight packing)
    \item Cis configuration (hydrogens same side) is natural form
    \item Sources: Olive oil (mono), fish, nuts, seeds (poly)
    \item Health effects: Lower LDL, may raise HDL; omega-3s reduce inflammation
  \end{itemize}
  
  \textbf{Trans Fats:}
  \begin{itemize}
    \item Structure: Unsaturated but with trans configuration (hydrogens opposite sides)
    \item Behave like saturated fats (straighter chain)
    \item Created by partial hydrogenation of vegetable oils
    \item Sources: Processed foods, margarine (historically), fried foods
    \item Health effects: Worst for cardiovascular health—raise LDL, lower HDL, increase inflammation; largely banned
  \end{itemize}
  
  \textbf{Dietary recommendations:} Replace saturated and trans fats with unsaturated fats. Emphasize omega-3 sources. Limit processed foods containing industrial trans fats.

\end{enumerate}

\end{document}