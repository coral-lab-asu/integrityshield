\documentclass[11pt]{article}

\usepackage[margin=1in]{geometry}
\usepackage{enumitem}
\usepackage{amsmath, amssymb}
\usepackage{hyperref}

\setlist[enumerate]{itemsep=0.8em, topsep=0.8em}

\begin{document}

\begin{center}
  {\Large \textbf{Organic Chemistry}}\\[0.25em]
  {\large \textbf{Functional Groups and Reaction Mechanisms Quiz -- Answer Key}}\\[0.5em]
\end{center}

\begin{enumerate}[label=\textbf{\arabic*.}]
  \item \textbf{(C) Carboxylic acid.} Carboxylic acids contain the --COOH group (carbonyl + hydroxyl). Aldehydes have C=O bonded to H, ketones have C=O between two carbons, and esters have C=O bonded to --OR.

  \item \textbf{(C) Concentration of both substrate and nucleophile.} S\textsubscript{N}2 is bimolecular with rate = k[substrate][nucleophile]. Both species participate in the rate-determining step.

  \item \textbf{(D) Tertiary carbocation.} Carbocation stability increases with substitution due to hyperconjugation and inductive electron donation: tertiary > secondary > primary > methyl.

  \item \textbf{(B) The middle carbon (C2).} Markovnikov's rule states that H adds to the carbon with more hydrogens, and Br adds to the more substituted carbon, forming the more stable carbocation intermediate.

  \item \textbf{(B) NaBH\textsubscript{4}.} Sodium borohydride is a mild reducing agent that reduces ketones and aldehydes to alcohols. KMnO\textsubscript{4} is an oxidizing agent, PCC oxidizes alcohols, and H\textsubscript{2}SO\textsubscript{4} is an acid catalyst.

  \item \textbf{True.} Enantiomers are non-superimposable mirror images with identical melting points, boiling points, and solubilities. They differ only in optical rotation direction (+/−).

  \item \textbf{False.} S\textsubscript{N}1 proceeds through a planar carbocation intermediate, allowing nucleophilic attack from either face, resulting in racemization (mixture of retention and inversion). S\textsubscript{N}2 proceeds with complete inversion (Walden inversion).

  \item \textbf{False.} Aromatic compounds are less reactive than alkenes toward electrophilic addition because addition would destroy the aromatic stabilization. Instead, they undergo electrophilic aromatic substitution to preserve aromaticity.

  \item \textbf{S\textsubscript{N}1 Mechanism:}
  \begin{itemize}
    \item Two-step process: (1) slow ionization to form carbocation, (2) fast nucleophilic attack
    \item Rate = k[substrate] (unimolecular)
    \item Favored by: tertiary substrates, weak nucleophiles, polar protic solvents (stabilize carbocation), good leaving groups
    \item Stereochemistry: racemization (planar carbocation)
  \end{itemize}
  
  \textbf{S\textsubscript{N}2 Mechanism:}
  \begin{itemize}
    \item One-step concerted process: nucleophile attacks as leaving group departs
    \item Rate = k[substrate][nucleophile] (bimolecular)
    \item Favored by: primary substrates (less steric hindrance), strong nucleophiles, polar aprotic solvents (don't solvate nucleophile), good leaving groups
    \item Stereochemistry: complete inversion (backside attack)
  \end{itemize}
  
  \textbf{Substrate effects:} Methyl/primary → S\textsubscript{N}2; tertiary → S\textsubscript{N}1; secondary → depends on other factors.

  \item \textbf{Aromaticity criteria (Hückel's rule):}
  \begin{itemize}
    \item Cyclic structure
    \item Planar (allows orbital overlap)
    \item Fully conjugated (p orbital on every atom in ring)
    \item Contains (4n + 2) π electrons (n = 0, 1, 2...)
  \end{itemize}
  
  \textbf{Benzene stability:} Benzene has 6 π electrons (n=1), satisfying Hückel's rule. The delocalized electrons create a continuous ring of electron density above and below the plane. Resonance energy (~36 kcal/mol) makes benzene ~36 kcal/mol more stable than hypothetical cyclohexatriene with localized double bonds.
  
  \textbf{Electrophilic aromatic substitution (EAS):}
  \begin{itemize}
    \item Mechanism: electrophile attacks π system → forms resonance-stabilized carbocation (arenium ion) → base removes H\textsuperscript{+} to restore aromaticity
    \item Substitution rather than addition preserves the aromatic ring
    \item Examples: halogenation, nitration, sulfonation, Friedel-Crafts reactions
  \end{itemize}
  
  \textbf{Conclusion:} Aromatic stability drives reactivity patterns—substitution preserves the (4n+2) π electron system while addition would destroy it.

\end{enumerate}

\end{document}