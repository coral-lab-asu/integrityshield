\documentclass[11pt]{article}

\usepackage[margin=1in]{geometry}
\usepackage{enumitem}
\usepackage{amsmath, amssymb}
\usepackage{hyperref}

\setlist[enumerate]{itemsep=0.8em, topsep=0.8em}
\setlist[itemize]{itemsep=0.3em, topsep=0.3em}

\begin{document}

\begin{center}
  {\Large \textbf{Organic Chemistry}}\\[0.25em]
  {\large \textbf{Functional Groups and Reaction Mechanisms Quiz}}\\[0.5em]
\end{center}

\noindent\textbf{Instructions:}
\begin{itemize}
  \item Answer all questions.
  \item For Questions 1--5, choose the best option.
  \item For Questions 6--8, mark True or False.
  \item For Questions 9--10, write detailed answers with mechanisms where appropriate.
\end{itemize}

\vspace{0.5em}

\begin{enumerate}[label=\textbf{\arabic*.}]
  \item Which functional group is characterized by a carbonyl group bonded to a hydroxyl group?
  \begin{enumerate}[label=(\Alph*)]
    \item Aldehyde
    \item Ketone
    \item Carboxylic acid
    \item Ester
  \end{enumerate}

  \item In an S\textsubscript{N}2 reaction, the rate depends on:
  \begin{enumerate}[label=(\Alph*)]
    \item Concentration of substrate only
    \item Concentration of nucleophile only
    \item Concentration of both substrate and nucleophile
    \item Temperature only
  \end{enumerate}

  \item Which of the following is the most stable carbocation?
  \begin{enumerate}[label=(\Alph*)]
    \item Methyl carbocation
    \item Primary carbocation
    \item Secondary carbocation
    \item Tertiary carbocation
  \end{enumerate}

  \item Markovnikov's rule predicts that in the addition of HBr to propene, the bromine will attach to:
  \begin{enumerate}[label=(\Alph*)]
    \item The terminal carbon (C1)
    \item The middle carbon (C2)
    \item Both carbons equally
    \item Neither carbon
  \end{enumerate}

  \item Which reagent is commonly used to reduce a ketone to a secondary alcohol?
  \begin{enumerate}[label=(\Alph*)]
    \item KMnO\textsubscript{4}
    \item NaBH\textsubscript{4}
    \item H\textsubscript{2}SO\textsubscript{4}
    \item PCC
  \end{enumerate}

  \item Enantiomers have identical physical properties except for the direction they rotate plane-polarized light. (True/False)

  \item An S\textsubscript{N}1 reaction proceeds with inversion of configuration at the stereocenter. (True/False)

  \item Aromatic compounds are more reactive than alkenes in electrophilic addition reactions. (True/False)

  \item Compare and contrast S\textsubscript{N}1 and S\textsubscript{N}2 nucleophilic substitution mechanisms. Discuss the factors that favor each pathway, including substrate structure, nucleophile strength, and solvent effects.
  \vspace{8em}

  \item Explain the concept of aromaticity and Hückel's rule. Why is benzene unusually stable compared to hypothetical cyclohexatriene? Describe how electrophilic aromatic substitution preserves aromaticity.
  \vspace{8em}

\end{enumerate}

\end{document}