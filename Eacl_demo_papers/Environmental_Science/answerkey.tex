\documentclass[11pt]{article}

\usepackage[margin=1in]{geometry}
\usepackage{enumitem}
\usepackage{amsmath, amssymb}
\usepackage{hyperref}

\setlist[enumerate]{itemsep=0.8em, topsep=0.8em}

\begin{document}

\begin{center}
  {\Large \textbf{Environmental Science}}\\[0.25em]
  {\large \textbf{Climate Change and Ecosystems Quiz -- Answer Key}}\\[0.5em]
\end{center}

\begin{enumerate}[label=\textbf{\arabic*.}]
  \item \textbf{(C) Carbon dioxide (CO$_2$).} While methane has higher warming potential per molecule, CO$_2$ contributes most to total anthropogenic warming due to its abundance and persistence in the atmosphere.

  \item \textbf{(B) A reservoir that absorbs more carbon than it releases.} Natural carbon sinks include forests, oceans, and soil. They remove CO$_2$ from the atmosphere through photosynthesis and dissolution.

  \item \textbf{(B) Absorption of atmospheric CO$_2$.} When CO$_2$ dissolves in seawater, it forms carbonic acid, lowering ocean pH. This threatens shell-forming organisms like corals and mollusks.

  \item \textbf{(B) Melting ice reducing Earth's albedo.} As reflective ice melts, darker ocean/land surfaces absorb more heat, causing further warming and more melting—a self-reinforcing cycle.

  \item \textbf{(B) 1.5°C to well below 2°C above pre-industrial levels.} The 2015 Paris Agreement set this target, with countries submitting nationally determined contributions (NDCs) to achieve it.

  \item \textbf{False.} The natural greenhouse effect is essential for life, keeping Earth approximately 33°C warmer than it would be otherwise. The concern is the enhanced greenhouse effect from human emissions.

  \item \textbf{True.} When ocean temperatures rise, corals expel their symbiotic zooxanthellae algae, losing their color and primary food source. Prolonged stress leads to coral death.

  \item \textbf{True.} Forests are major carbon sinks. Deforestation releases stored carbon (through burning/decomposition) and eliminates future sequestration capacity, contributing roughly 10\% of global emissions.

  \item \textbf{Natural carbon cycle:} Carbon moves between atmosphere, biosphere, oceans, and geosphere. Plants absorb CO$_2$ through photosynthesis; respiration and decomposition return it to the atmosphere. Oceans absorb and release CO$_2$. Geological processes store carbon in rocks and fossil fuels over millions of years.
  
  \textbf{Human disruption:} Humans have increased atmospheric CO$_2$ from ~280 ppm (pre-industrial) to over 420 ppm by releasing stored carbon faster than natural sinks can absorb it.
  
  \textbf{Major anthropogenic sources:}
  
  (1) \textit{Fossil fuel combustion:} Burning coal, oil, and natural gas for energy releases ~35 billion tonnes CO$_2$ annually—the largest source.
  
  (2) \textit{Deforestation and land use change:} Clearing forests for agriculture releases stored carbon and reduces sequestration capacity.
  
  (3) \textit{Industrial processes:} Cement production releases CO$_2$ from limestone heating; steel and chemical manufacturing also contribute significantly.
  
  (4) \textit{Agriculture:} Livestock produce methane; rice paddies and fertilizers release methane and nitrous oxide.

  \item \textbf{Climate impacts on biodiversity:}
  
  \textit{Habitat shifts:} Species ranges are moving poleward and to higher elevations. Example: Many butterfly and bird species in Europe have shifted northward.
  
  \textit{Phenological mismatches:} Timing of life events (migration, breeding, flowering) becomes desynchronized. Example: Great tits breeding earlier than caterpillar peak emergence, reducing chick survival.
  
  \textit{Coral reef ecosystems:} Rising temperatures and acidification threaten reefs. The Great Barrier Reef has experienced mass bleaching events, affecting thousands of dependent species.
  
  \textit{Arctic ecosystems:} Polar bears, walruses, and seals face habitat loss as sea ice diminishes.
  
  \textit{Extinction risk:} Studies estimate 15-37\% of species face extinction risk from climate change by 2050.
  
  \textbf{Conservation strategies:} Protected area networks with connectivity corridors; assisted migration for vulnerable species; reducing non-climate stressors (habitat fragmentation, pollution); ex-situ conservation (seed banks, captive breeding); ecosystem-based adaptation.

\end{enumerate}

\end{document}